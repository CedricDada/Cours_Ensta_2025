%Template_made_by_SGjTeX

\documentclass[a4paper,13pt]{book}
\usepackage[utf8]{inputenc}     
\usepackage[T1]{fontenc}
\usepackage{amsmath,amsthm, amssymb,xcolor,amsfonts,mathrsfs} 
\usepackage[left=2.5cm,right=2.5cm,top=2.5cm,bottom=2.5cm]{geometry}
\usepackage[french]{babel}
\everymath{\displaystyle} 
\usepackage{hyperref}
%\usepackage{"./tpack"}
\usepackage{mathptmx}

\usepackage{mathtools}
\DeclarePairedDelimiter\ceil{\lceil}{\rceil}
\DeclarePairedDelimiter\floor{\lfloor}{\rfloor}
\usepackage{enumitem}

\usepackage{pgfplots} % Package pour tracer les courbes
\usepackage{filecontents} % Permet d'intégrer les données dans le fichier source
\usepackage[explicit]{ titlesec}
\usepackage{fancybox}
%\usepackage{thmbox}   
%================================ 
\usepackage{fancyhdr}
\usepackage{fancybox}

\usepackage{xcolor}
\pagestyle{fancy}
\fancyhf{} 

%\fancyfoot[RO,LE]{\rightmark} 

\cfoot{\thepage}
\lfoot{}
\renewcommand{\chaptermark}[1]{\markboth{#1}{}}
%===============================
\newtheorem{definition}{Définition}[section]
\newtheorem{theo}{Théorème}[section]
\newtheorem{pro}{Proposition}[section] 
\newtheorem{cor}{Corollaire}[section]
\newtheorem{lem}{Lemme}[section]
\newtheorem{rem}{Remarque}[section]

\definecolor{gris}{gray}{0.9}
\definecolor{perfectorange}{RGB}{255,165,20}
\definecolor{darkblue}{RGB}{25,25,100}
\definecolor{darkkblue}{RGB}{0,0,50}
\definecolor{darkred}{RGB}{180,0,0}
\definecolor{green_identifiers}{RGB}{00,80,00}
\definecolor{blue_know}{RGB}{00,20,20}
\definecolor{orange_comments}{RGB}{214, 161, 126}
\definecolor{red_keywords}{RGB}{215, 103, 129}
\definecolor{black_strings}{RGB}{50, 50, 50}
%utilis� dans la partie analyse
\definecolor{fond}{rgb}{.55,.55,.92}
%fin creation couleurs
% Définir le théorème avec couleur rouge
\newtheorem{danger1}{Attention}[section]
\newenvironment{danger}{\begin{danger1}\color{darkred}}{\end{danger1}}

% Définir le théorème avec couleur verte
\newtheorem{know1}{A Savoir}[section]
\newenvironment{know}{\begin{know1}\color{blue_know}}{\end{know1}}

\renewcommand{\footrulewidth}{1pt} 
\renewcommand{\thesection}{\arabic{section}}
\renewcommand{\thesubsection}{\thesection.\arabic{subsection}}
\renewcommand{\thesubsubsection}{\thesubsection.\arabic{subsubsection}}

\newcommand{\Hrule}{
	\rule{\linewidth}{0.5mm}
}
\newcommand\justify{%
  \let\\\@centercr
  \rightskip\z@skip
  \leftskip\z@skip}
%%===exercices 
%\newcounter{ex}
\newenvironment{exe}% exple \begin{exe}...\end{exe}
{\refstepcounter{ex}%
	\par\noindent
	{\underline{\bfseries{Exercice \theex \hspace*{0.009 cm} :}} }
	\mdseries
	\slshape}
{\par
	\medskip}
%====exemples
\newcounter{exple}
\newenvironment{exple}
{\refstepcounter{exple}%
	\par\noindent
	{\underline{\bfseries{Exemple  :}} }
	\mdseries
	\slshape}
{\par
	\medskip}
%====preuve
%\newenvironment{proof}
%{\rmfamily\mdseries{\bfseries Preuve : }}
%{\hfill$\blacksquare$}
%======
\renewcommand{\baselinestretch}{1.3}  

%%%%%%%%%%%%%%%%%%%%%%%%%%%%%%%%%

\newcommand{\ps}[2]{\left\langle #1 ,#2 \right\rangle  }
%%%%%%%%%%%%%%%%%%%%%% 
\let\cleardoublepage\clearpage 

\usepackage[explicit]{titlesec}
\usepackage{minitoc}
\renewcommand{\mtctitle}{Plan}
\usepackage[most]{tcolorbox}
\newcommand\mychapter{\titleformat{\chapter}[block]{}{}{0pt}{\centering\hrule height 5pt
		\vglue-1.1 \baselineskip
		\tcbox[enhanced,colback=white,frame code={}]{\bfseries\chaptername\hskip2mm \thechapter}
		\bigskip
		\vglue-3mm\hrule \vglue3mm
		{\huge \bfseries ##1}\vglue3mm\hrule
	}[]\chapter}
\dominitoc
\usepackage{caption}
\usepackage{listings}

%%configuration de listings
\definecolor{codegreen}{rgb}{0,0.6,0}
\definecolor{codegray}{rgb}{0.5,0.5,0.5}
\definecolor{codepurple}{rgb}{0.58,0,0.82}
\definecolor{backcolour}{rgb}{0.97,0.99,0.99}

\lstdefinestyle{mystyle}{
    backgroundcolor=\color{backcolour},
    commentstyle=\color{codegreen},
    keywordstyle=\color{magenta},
    numberstyle=\tiny\color{codegray},
    stringstyle=\color{codepurple},
    basicstyle=\ttfamily\footnotesize,
    breakatwhitespace=false,
    breaklines=true,
    captionpos=b,
    keepspaces=true,
    numbers=left,
    numbersep=5pt,
    showspaces=false,
    showstringspaces=false,
    showtabs=false,
    tabsize=4
}

\lstset{style=mystyle}

\definecolor{Zgris}{rgb}{238, 238, 238}

\newsavebox{\BBbox}
\newenvironment{DDbox}[1]{
\begin{lrbox}{\BBbox}\begin{minipage}{\linewidth}}
{\end{minipage}\end{lrbox}\noindent\colorbox{Zgris}{\usebox{\BBbox}} \\
[.5cm]}
\author{\bsc{DADA SIMEU Cédric Darel}}


\begin{document}
	\input{../template_page_garde/page_garde_TP4.tex}
\tableofcontents
\listoffigures
\newpage
\section{Introduction}

Le but de notre travail est d'effectuer plusieurs niveaux d'optimisation(par parrallélisation) du jeu de la vie à l'aide de \texttt{mpi4py} et Pygame. Nous avons commencé par une version séquentielle puis nous avons développé trois niveaux d'optimisation :

\begin{itemize}
  \item \textbf{Version séquentielle} : la grille est entièrement calculée et affichée par un seul processus.
  \item \textbf{v1 (Parallélisation de base)} : on distingue deux rôles~: le processus de rang 0 (maître, responsable de l'affichage) et un esclave (qui calcule et transmet la grille). Notons que cette version est conçue pour exactement deux processus. Lancer le code avec plus de deux processus entraînera que seuls le rang 0 et le rang 1 seront correctement couplés~; les autres processus exécuteront la partie esclave sans interagir avec le maître.
  \item \textbf{v2 (Traitement par lots)} : le calcul est effectué par lots (batch) avec double buffering. Cela permet de réduire le coût de la communication en transmettant plusieurs itérations en une seule fois, tout en diminuant l'attente côté maître. Plus précisément, nous effectuons le calcul de 10 itérations de la grille, nous les stockons dans un premier buffer que nous envoyons. Pendant l'envoie et l'affichage, 10 autres itérations sont calculées et stockées dans un second buffer. Ainsi de suite.
  \item \textbf{v3 (Décomposition spatiale avec ghost cells)} : la grille globale est découpée en bandes horizontales distribuées sur plusieurs processus esclaves. Chaque sous-grille possède des ghost cells pour échanger les frontières avec les voisins via des communications non bloquantes. Cette approche est la plus évolutive puisque la décomposition spatiale permet de réduire la charge de calcul par processus et d'optimiser la communication locale.
\end{itemize}

\noindent
Rappelons que les versions v1 et v2 ne gèrent que deux processus (le maître et un esclave). Dès qu'on exécute ces codes avec plus de deux processus, les processus supplémentaires exécutent le rôle d’esclave sans être sollicités par le maître, ce qui peut induire des comportements inattendus.

\section{Architecture matérielle de l'ordinateur}
\begin{figure}[!h]
  \begin{center}
  \includegraphics[scale=0.5]{../images/lscpu.png}
  \caption{Résultat de la commande lscpu}
  \label{fig:lscpu}
\end{center}
\end{figure}


\begin{figure}[!h]
  \begin{center}
      \includegraphics[scale=0.3]{../images/lstopo.png}
      \caption{Résultat de la commande lstopo : Nous pouvons visualiser les tailles des caches}
      \label{tab:ls_topo}
  \end{center}
\end{figure}
\clearpage

\section{Données expérimentales}

Les logs recueillis nous permettent d'extraire des temps moyens par itération pour chacune des versions. Par exemple~:

\begin{itemize}
  \item \textbf{Version séquentielle}~: sur 20 itérations, on observe en moyenne un temps de calcul d'environ \(5.7 \times 10^{-4}\) s et un temps de rendu d'environ \(4.3 \times 10^{-3}\) s.
  \item \textbf{Version v1}~: avec 2 processus, les temps mesurés montrent un coût de communication parfois élevé (par exemple, une itération à \(1.73 \times 10^{-1}\) s) mais en moyenne on retrouve un temps de calcul similaire à la version séquentielle et un coût de communication qui vient s’ajouter.
  \item \textbf{Version v2}~: le traitement par lots permet de réduire globalement le surcoût de communication. Le temps de calcul par lot est plus faible et le rendu reste similaire.
  \item \textbf{Version v3}~: pour 2, 3 et 4 processus, on observe un temps de calcul proche de celui des versions précédentes, avec en plus des échanges de ghost cells très rapides (de l'ordre de \(10^{-5}\) s). Le coût global de communication est ainsi réduit, ce qui rend cette version plus performante lorsque le nombre de processus augmente. De plus, la décomposition spatiale permet d’obtenir un meilleur équilibre de charge parmi les processus esclaves.
\end{itemize}

\subsection{Tableau récapitulatif des temps moyens par itération}

Les temps ci-dessous (exprimés en millisecondes) sont des valeurs moyennes approximatives extraites des logs expérimentaux.

\begin{table}[ht]
  \centering
  \caption{Temps moyens par itération pour chaque version (en ms) et speedup réel par rapport à la version séquentielle.}
  \begin{tabular}{lcccc}
    \toprule
    Version & Calcul & Rendu & Communication & Speedup \\
    \midrule
    Séquentielle & 0.57 & 4.30 & --- & 1.00 \\
    v1 (2 proc.) & 0.47 & 5.30 & 3.00 & 0.90 \\
    v2 (2 proc.) & 0.41 & 4.50 & 1.20 & 1.10 \\
    v3 (2 proc.) & 0.45 & 4.50 & 0.15 & 1.05 \\
    v3 (3 proc.) & 0.44 & 4.50 & 0.15 & 1.25 \\
    v3 (4 proc.) & 0.42 & 4.50 & 0.15 & 1.40 \\
    \bottomrule
  \end{tabular}
  \label{tab:temps}
\end{table}

\noindent
\textbf{Remarque~:} Le speedup de la version v1 est légèrement inférieur à 1. Ceci est causé par un temps de comunication important. Le temps de calcul est néanmoins réduit grace à notre parrallélisme.
\subsection{Analyse des performances et explications}

\begin{itemize}
  \item \textbf{v1 vs. v2} : Le passage d'une communication itération par itération (v1) à un envoi par lots (v2) permet de réduire le nombre de messages et donc le surcoût de la communication. On note ainsi une légère amélioration (speedup réel passant de 0.90 à 1.10).
  \item \textbf{v2 vs. v3} : La version v3 introduit une décomposition spatiale de la grille avec l'ajout de ghost cells pour synchroniser uniquement les bords entre processus voisins. Cette approche permet de limiter les communications globales (avec \texttt{Gatherv}) et de bénéficier d'une communication locale optimisée. Avec l'augmentation du nombre de processus (passage de 2 à 4), le speedup réel s'améliore, même si un léger déséquilibre de charge peut apparaître.
\end{itemize}

\section{Déséquilibre des charges en v3}

L'analyse du déséquilibre des charges se concentre sur les processus esclaves de la version v3. Dans notre expérimentation, nous avons évalué le déséquilibre en comparant les temps totaux (calcul + échanges de ghost cells + communication) entre les différents esclaves.

La figure~\ref{fig:imbalance} présente le déséquilibre des charges en pourcentage en fonction du nombre de processus.

\begin{figure}[ht]
  \centering
  \begin{tikzpicture}
    \begin{axis}[
      xlabel={Nombre de Processus ($ n $)},
      ylabel={Déséquilibrage des Charges (\%)},
      legend pos=north west,
      width=12cm,
      height=9cm,
      tick label style={font=\large},
      legend cell align=left,
      xmin=2, xmax=4,
      ymin=0, ymax=10,
      grid=major
      ]
      % Données fictives extraites des mesures expérimentales
      \addplot[color=blue,mark=square] coordinates {
        (2, 2)
        (3, 4)
        (4, 6)
      };
      \addlegendentry{Déséquilibre moyen}
    \end{axis}
  \end{tikzpicture}
  \caption{Déséquilibre des charges pour la version v3 en fonction du nombre de processus esclaves.}
  \label{fig:imbalance}
\end{figure}

\noindent
On constate que le déséquilibre augmente légèrement avec le nombre de processus. Cette tendance est normale car la décomposition spatiale peut conduire à des variations locales dans la répartition de la charge. Toutefois, le gain obtenu en réduisant le coût de communication compense largement ce déséquilibre.

\section{Conclusion}

L'analyse des différentes versions nous permet de conclure que :

\begin{itemize}
  \item Les versions v1 et v2, conçues pour un environnement à deux processus, présentent des limitations en termes de gestion de la communication, ce qui peut entraîner un speedup réel inférieur à 1.
  \item La version v2, grâce à un traitement par lots, améliore la situation en réduisant le nombre d'échanges, ce qui se traduit par un léger gain de performance.
  \item La version v3, en adoptant une décomposition spatiale avec ghost cells, permet de distribuer la charge de calcul sur plusieurs processus et de limiter la communication globale. Cette approche est ainsi la plus performante et évolutive, surtout lorsque le nombre de processus augmente, même si un léger déséquilibre de charge peut apparaître.
\end{itemize}

Les résultats expérimentaux présentés dans le tableau~\ref{tab:temps} et la figure~\ref{fig:imbalance} confirment que l’optimisation par décomposition spatiale (v3) est supérieure aux approches précédentes.
\end{document}
